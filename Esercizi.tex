\chapter{Esercizi}
\sisetup{%
range-phrase = {\ \linebreak[0]\text{to}\ \nolinebreak},
list-separator = {\text{, }},
list-final-separator = {,\ \linebreak[0]\text{e }},
list-pair-separator = {\ \text{and}\ },
list-separator = {,\ \linebreak[0]}
}%
\section{Frequenze}
\begin{esercizio}\label{EsercizioUno} 
Durante una misurazione vengono riportate le seguenti altezze \SIlist{164; 151; 170;  150; 169; 172; 166; 178; 156; 172; 174; 161; 168; 160; 156; 179;180; 172; 152; 172; 175; 166; 174; 150; 169; 150; 169}{\cm}. Determinarne  le frequenze relative e percentuali.\index{Frequenza!relativa}\index{Frequenza!percentuale}
\end{esercizio}
\begin{soluzione}
	Per risolvere l'esercizio bisogna costruire la tabella delle frequenze assolute elencando le altezze e riportando nella colonna accanto quante volte quell'altezza compare nell'elenco.  Si ottiene la tabella~\vref{Tab:EsercizioUnoA}. Dopo avere calcolato la somma delle frequenze, possiamo costruire la tabella delle frequenze relative. In pratica aggiungiamo una colonna alla tabella delle frequenze assolute. Ogni riga di questa colonna è uguale alla frequenza assoluta corrispondente diviso la somma delle frequenze relative. A lavoro ultimato abbiamo la tabella~\vref{Tab:EsercizioUnoB}. Da quello che abbiamo appena costruito otteniamo -85999le frequenze percentuali 
	\begin{table}
		\begin{subtable}[b]{.2\linewidth}
		\centering
	\begin{tabular}{l>{\xstrut$}l<{$}}
		\toprule
		x & n \\
		\midrule
		150 & 1 \\
		151 & 1 \\
		152 & 1 \\
		156 & 2 \\
		160 & 1 \\
		161 & 1 \\
		164 & 1 \\
		166 & 2 \\
		168 & 1 \\
		169 & 2 \\
		170 & 1 \\
		172 & 3 \\
		174 & 2 \\
		175 & 1 \\
		178 & 1 \\
		179 & 1 \\
		180 & 1 \\
		\midrule
		Tot.&24\\
		\bottomrule
	\end{tabular}
\subcaption{Frequenze assolute}\label{Tab:EsercizioUnoA}
	\end{subtable}
	\begin{subtable}[b]{.3\linewidth}
	\centering
	\begin{tabular}{ll>{\xstrut$}l<{$}}
		\toprule
		x & n & Fr\\
		\midrule
		150 & 1 & \frac{1}{24} \\
		151 &1&\frac{1}{24} \\
		152 &1&\frac{1}{24} \\
		156 &2&\frac{2}{24} \\
		160 &1&\frac{1}{24} \\
		161 &1&\frac{1}{24} \\
		164 &1&\frac{1}{24} \\
		166 &2&\frac{2}{24} \\
		168 &1&\frac{1}{24} \\
		169 &2&\frac{2}{24} \\
		170 &1&\frac{1}{24} \\
		172 &3&\frac{3}{24} \\
		174 &2&\frac{2}{24} \\
		175 &1&\frac{1}{24} \\
		178 &1&\frac{1}{24} \\
		179 &1&\frac{1}{24} \\
		180 &1&\frac{1}{24} \\
		\midrule
		Tot.&24&1\\
		\bottomrule
	\end{tabular}
\subcaption{Frequenze relative}\label{Tab:EsercizioUnoB}
\end{subtable}
	\begin{subtable}[b]{.4\linewidth}
		\centering
	\begin{tabular}{ll>{\xstrut$}l <{$}l}
		\toprule
		x & n & Fr&F\%\\
		\midrule
		150 &1&\frac{1}{24}&\MyNum{4.166666667} \\
		151 &1&\frac{1}{24}&\MyNum{4.166666667} \\
		152 &1&\frac{1}{24}&\MyNum{4.166666667} \\
		156 &2&\frac{2}{24}&\MyNum{8.333333333} \\
		160 &1&\frac{1}{24}&\MyNum{4.166666667} \\
		161 &1&\frac{1}{24}&\MyNum{4.166666667} \\
		164 &1&\frac{1}{24}&\MyNum{4.166666667} \\
		166 &2&\frac{2}{24}&\MyNum{8.333333333} \\
		168 &1&\frac{1}{24}&\MyNum{4.166666667} \\
		169 &2&\frac{2}{24}&\MyNum{8.333333333} \\
		170 &1&\frac{1}{24}&\MyNum{4.166666667} \\
		172 &3&\frac{3}{24}&\MyNum{12.5} \\
		174 &2&\frac{2}{24}&\MyNum{8.333333333} \\
		175 &1&\frac{1}{24}&\MyNum{4.166666667} \\
		178 &1&\frac{1}{24}&\MyNum{4.166666667} \\
		179 &1&\frac{1}{24}&\MyNum{4.166666667} \\
		180 &1&\frac{1}{24}&\MyNum{4.166666667} \\
		\midrule
		Tot.&24&1&100\%\\
		\bottomrule
	\end{tabular}
	\subcaption{Frequenze percentuali}\label{Tab:EsercizioUnoC}
\end{subtable}
	\captionsetup{labelformat=empty}
		\caption{Esercizio~\ref{EsercizioUno}}
		\label{tab:tabellaEsercizioUno}
	\end{table}
\end{soluzione}
\begin{esercizio}\label{EsercizioDue} 
 Due classi effettuano la medesima prova. La prima ottiene i seguenti voti: \numlist{3; 3; 4; 2; 2; 7; 6; 1; 6; 2; 3; 8; 8; 6; 1; 7; 7; 7; 8; 1; 9; 8; 3; 7; 7; 8; 5}, la seconda \numlist{9; 7; 4; 3; 6; 5; 3; 4; 2; 7; 6; 6; 1; 4; 5; 2; 2; 2; 1; 8; 8; 2; 7; 6; 3; 10; 7; 6; 9;
	5; 6; 6; 4}. Confrontare le frequenze percentuali delle due prove.
\end{esercizio}\index{Frequenza!percentuale}
\begin{soluzione}
	Per risolvere questo esercizio bisogna costruire, per ogni classe la tabella delle frequenza percentuali. Iniziamo dalla prima classe. 
	\begin{table}
\begin{tabular}{cccc}
	\toprule
V	& n & F\% &  \\
1	&  &  &  \\
2	&  &  &  \\
3	&  &  &  \\
4	&  &  &  \\
5	&  &  &  \\
6	&  &  &  \\
7	&  &  &  \\
8	&  &  &  \\
9	&  &  &  \\
10	&  &  &  \\
	\midrule
Tot.&  &  &  \\
	\bottomrule
\end{tabular}		
\captionsetup{labelformat=empty}
\caption{Esercizio~\ref{EsercizioDue}}
\label{tab:tabellaEsercizioDue}
	\end{table}
\end{soluzione}
\begin{esercizio}
Dato i seguenti dati \numlist{25; 18; 15; 17; 6; 22; 24; 13; 22; 12; 6; 18; 24; 18; 7; 6; 15; 10; 8; 21; 15; 13;
	24; 25; 14; 16; 7; 23; 15; 15; 18; 7; 23 8; 20; 19; 11; 16; 15 15; 17; 11; 12; 14; 11
	21; 18; 20; 23; 11; 9} determinarne la frequenza cumulata.
\end{esercizio}\index{Frequenza!cumulata}
\begin{esercizio}
	Due classi effettuano la medesima prova. La prima ottiene i seguenti voti: \numlist{4; 8; 5; 4; 6; 3; 2; 1; 5; 8; 3; 3; 3; 3; 8; 4; 3; 7;}, la seconda \numlist{6; 9; 3; 2; 4; 10; 5; 4; 8; 8; 8; 5; 7; 1; 9; 7; 7; 1; 8; 9; 5; 10; 3; 4; 1; 6; 2; 3; 5;
		8; 2; 7; 8 8; 5; 10}. Confrontare le frequenze percentuali delle due prove.
\end{esercizio}\index{Frequenza!percentuale} 
\section{Medie, mode, scarti}
\begin{esercizio}
	Calcolare lo scarto di questi numero tra il $14$ e i numeri seguenti \numlist{15; 12; 11; 14; 15; 15; 12; 12; 18; 13; 13; 13; 14; 15; 16; 17; 13; 13}
\end{esercizio}\index{Scarto}
\begin{esercizio}
	Dopo aver calcolato la media di questi numeri \numlist{12; 20; 8; 10; 8; 16; 11; 9; 13; 15; 13; 20; 15; 20; 8; 19; 15; 15; 10; 10; 10; 8;
		10; 10; 12; 11; 14; 19; 19; 7; 19; 9; 17 8; 9; 9; 14; 7; 13 11; 18; 12; 13; 17; 9 9; 5;
		9; 10; 7; 13}, calcolare lo scarto di questi valori da essa.
\end{esercizio}\index{Media}\index{Scarto}
\begin{esercizio}
	 Date le seguenti altezze \SIlist{1.88; 1.88; 1.61; 1.89; 1.74; 1.64; 2; 1.84; 1.95; 1.73; 1.80; 1.96; 1.77; 1.90; 1.76; 1.71; 1.90;
		1.92; 1.91; 1.89; 1.64; 1.81; 2; 1.64; 1.74; 1.67; 1.89; 1.95; 1.66; 1.96; 1.82; 1.94}{\m} calcolarne media e scarto quadratico medio.
\end{esercizio}\index{Media}\index{Scarto!quadratico}
\begin{esercizio}
Dati i seguenti lanci di dadi, \numlist{1; 1; 5; 6; 3; 1; 2; 5; 2; 3; 2; 2; 1; 6; 1; 3; 4; 4; 1; 6; 1; 6; 3; 6; 6; 4; 4; 1; 5; 5}, calcolarne moda, mediana media.
\end{esercizio}\index{Moda}\index{Media}\index{Mediana}
\begin{esercizio}
	
	\item Viene fatto un sondaggio, viene chiesto di esprime un giudizio che varia da zero pessimo a quattro  ottimo. La tabella esprime il sondaggio
	\begin{center}
		\begin{tabular}{l*{5} {S[table-format=3.0]}}
			{Giudizio}	&0  &1  &2  &3&4  \\
			\midrule 
			{Frequenze}	& 100 &120  & 130 & 110&105 \\ 
		\end{tabular}
	\end{center} Calcolare media, mediana, moda, varianza, deviazione standard.
\end{esercizio}

